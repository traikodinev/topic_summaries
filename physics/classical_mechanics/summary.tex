\iffalse
This is a big-ish template that I've been using for years now.
A lot of the includes are redundant, so if you're reading this and have no idea why
I've included most things, don't worry - I don't either.
Also, compiling this (to PDF, for instance) relies on CSS files present on my
computer (coming from the packages). So it might not look as nice if you
do it on yours (especially if you don't have all the packages).

Ya'll've been warned.
\fi

\documentclass{article}
    \usepackage{subcaption}
    \usepackage{todonotes}
    \usepackage{amsmath}
    \usepackage{amssymb}
    \usepackage{color}
    % \usepackage[dvipsnames]{xcolor}
    \usepackage{graphicx}
    \usepackage{caption}
    \usepackage{float}
    \usepackage[hidelinks]{hyperref}
    \usepackage{enumitem}
    \usepackage[bottom]{footmisc}
    \usepackage{flexisym}
    \usepackage{cancel}
    \usepackage[braket]{qcircuit}
    \usepackage[margin=.8in, tmargin=.3in]{geometry}
    \renewcommand{\baselinestretch}{1.2}
    \newcommand{\eps}{\epsilon}


    \newcommand{\vq}{\mathbf{q}}
    \newcommand{\vqdot}{\mathbf{\dot{q}}}
    \newcommand{\vqddot}{\mathbf{\ddot{q}}}
    \newcommand{\vx}{\mathbf{x}}
    \newcommand{\vy}{\mathbf{y}}
    \newcommand{\vystar}{\mathbf{y}^*}
    \newcommand{\vu}{\mathbf{u}}
    \newcommand{\vv}{\mathbf{v}}
    \newcommand{\vf}{\mathbf{f}}
    \newcommand{\vr}{\mathbf{r}}
    \newcommand{\vp}{\mathbf{p}}
    \newcommand{\vg}{\mathbf{G}}
    \newcommand{\bM}{\mathbf{M}}
    \newcommand{\bC}{\mathbf{C}}
    \newcommand{\bS}{\mathbf{S}}
    \newcommand{\bJ}{\mathbf{J}}
    \newcommand{\vlam}{\pmb{\lambda}}
    \newcommand{\vxdot}{\mathbf{\dot{x}}}
    \newcommand{\argmin}{\operatornamewithlimits{arg\ min}}

    % \newcommand{\tens}{\otimes}


\newcommand{\tens}[1]{%
  \mathbin{\mathop{\otimes}\limits_{#1}}%
}

    \setlength{\parskip}{\baselineskip}
    % \newcommand{\l}{\left(}
    
    \usepackage{titlesec}
    \usepackage{physics}
    
    \setlength\parindent{0pt}
    \captionsetup{justification=centering}
    
    \title{Classical Mechanics: From Newton to Euler-Lagrange}
    \date{\today}
    \author{Traiko Dinev \textless traiko.dinev@gmail.com\textgreater}

\begin{document}
\maketitle
\textit{NOTE: These notes are a summary of Classical Mechanics: A Theoretical Minimum. They also contain some of David Morrin. Some of the notebooks are exercises I did through my own research.}

\textit{NOTE: Note this "summary" is NOT a reproduction of the course materials nor is it copied from the corresponding courses. It was entirely written and typeset from scratch.}

\textit{License: Creative Commons public license; See README.md of repository}

Hey, it is me, the author of these notes. I used to think reading physics was pointless practically speaking, so I did it a bit sparingly. But then.. I actually needed it in my own research! So I'm citing my work here~\cite{dinev2020modeling}. Hopefully it encourages more academics to read physics at a moderate to advanced level, as you never know when you'll need it!


\section{Lagrange}
\todo[inline]{format }

We use the Lagrangian method to derive the dynamics of the system ~\cite[Chapter~6]{morin_introduction_2008}. 
First, we define the position $x_b, z_b$ of the mass $m_b$ and its velocity $\dot{x}_b, \dot{z}_b$:
% 
% \vskip 0.1in
\begin{align}
    x_b &= x + l\ \sin(\theta) \\
    z_b &= z + l\ \cos(\theta) \nonumber \\
    \dot{x}_b &= \dot{x} + \dot{l}\ \sin(\theta) +
        l\ cos(\theta)\ \dot{\theta} \nonumber \\
    \dot{z}_b &= \dot{z} + \dot{l}\ \cos(\theta) -
        l\ sin(\theta)\ \dot{\theta} \nonumber
\end{align}
% \vskip 0.1in
% 
Lagrange's method states that for a system with total kinetic energy $T$ and potential energy $U$:
% 
\vskip 0.1in
\begin{equation} \label{eq:lagrange}
    \frac{d}{dt} \frac{\partial \mathcal{L}}{\partial \vqdot_i} - \frac{\partial \mathcal{L}}{\partial \vq_i} = \vf_{ext},
\end{equation}
\vskip 0.1in
% 
where $\mathcal{L} = T - U$ is the system's Lagrangian and $\vf_{ext}$ are external forces applied to the system. We now need to compute the system's kinetic and potential energy.

In general, every link will have a rotational and translational kinetic energy component. For the wheel we include a rotational kinetic energy term $I_w \dot{\phi}^2$ where $I_w = m_w R_w^2$ is the moment of inertia of the wheel. Since the point mass has a zero moment of inertia, it only has a translational kinetic energy $m_b \vv_b^T \vv_b$, where $\vv_b$ is the velocity of the point mass.

The only potential energy component is due to gravity $g$ acting on the wheel and the point mass. This leads to:
% 
\begin{align} \label{eq:energy1}
    T &= \frac{1}{2}(I_w \ \dot{\phi}^2 + m_w \dot{x}^2 +
        m_w \dot{z}^2 + m_b \dot{x}_b^2 + m_b \dot{z}_b^2) \\
    U &= m_w g z + m_b g z_b \label{eq:energy2} \\
    \mathcal{L} &= T - U \label{eq:energy3}
\end{align}

\bibliographystyle{IEEEtran}
\bibliography{IEEEabrv,IEEEconf,bib}
\end{document}
